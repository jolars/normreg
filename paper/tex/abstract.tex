Many regularized methods, such as the lasso and ridge regression, are sensitive to the
scales of the features in the data. As a consequence, it has become standard practice to
normalize (center and scale) features before fitting the models. There are, however, many
different ways to normalize the features and, as we show in this paper, which one you use
may have dramatic effects on the estimated coefficients from the resulting model. For the
lasso, for instance, entirely different sets of features may be selected depending on which
type of normalization is used. In spite of this, the interplay between normalization and
regularization has not been studied previously. In this paper, we begin to fill this
knowledge gap by studying binary and normally distributed features in the context of
regression with lasso, ridge, or elastic net regularization. We show that the class
balances of binary features has a direct relationship with their corresponding coefficients
and that these effects depend on the type regularization (lasso or ridge) used. We suggest
possible remedies for this problem and also discuss the case when data is mixed, that is,
contains both continuous and binary features.
