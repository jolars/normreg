\section{Preliminaries}

Throughout this paper, we assume that the data is generated from a linear model, \(y_i =
\beta_0^* + \vec x_i^\T \vec\beta^* + \varepsilon_i\) for \(i \in [n]\) where \([n] =
\{1,2,\dots,n\}\) and we use \(\beta_0^*\) and \(\vec\beta^*\) to denote the true intercept
and coefficients, respectively, and \(\varepsilon_i\) to denote measurement noise. \(\mat
X\) is the \(n \times p\) design matrix with features \(\vec x_j\) and \(\vec y\) the \(n
\times 1\) response vector. Furthermore, we use \(\hat\beta_0\) and \(\hat{\vec{\beta}}\)
to denote our estimates of the intercept and coefficients. Unless we state otherwise, we
assume \(\mat{X}\), \(\beta_0^*\), and \(\vec{\beta}^*\) to be fixed.%

There is much ambiguity regarding key terms in the field of normalization. \emph{Scaling},
\emph{standardization}, and \emph{normalization} are for instance used interchangeably
throughout the literature. Here, we define \emph{normalization} as the process of centering
and scaling the feature matrix, which we now formalize.

\begin{definition}[Normalization]
  \label{def:normalization}
  % Let \(\mat S\) be the \emph{scaling matrix}, which is a \(p \times p\) diagonal matrix with entries \(s_1, s_2, \dots, s_p\). Let \(\mat C\) be the \emph{centering matrix}, which is an \(n \times p\) matrix with each row equal to \([c_1, c_2, c_n]^\T\). Then the \emph{normalized design matrix} \(\tilde{\mat X}\) is defined as \(\tilde{\mat X} = (\mat X - \mat C)\mat S^{-1}\).
  Let \(\tilde{\mat X}\) be the normalized feature matrix, with elements \(\tilde{x}_{ij} =
  (x_{ij} - c_{j})/s_j,\) where \(x_{ij}\) is an element of the (unnormalized) feature matrix
  \(\mat{X}\) and \(c_j\) and \(s_j\) are the \emph{centering} and \emph{scaling} factors
  respectively.
\end{definition}

Some authors refer to the procedure in \Cref{def:normalization} as \emph{standardization},
but here we define standardization only as the case when centering with the mean and
scaling with the (uncorrected) standard deviation.

\subsection{Types of Normalization}

There are many different strategies for normalizing the design matrix. We list a few common
choices in \Cref{tab:normalization-types}. Standardization is perhaps the most popular type
of normalization, at least in the field of statistics. One of its benefits is that it
simplifies certain aspects of fitting the model, such as fitting the intercept. The
downside of standardization is that it involves centering by the mean, which destroys
sparsity. This is not a problem when \(\bm{X}\) is stored as a dense matrix; but when the
data is sparse, it may increase memory use and processing time.

% TODO: move to appendix?
\begin{table}[hbt]
  \centering
  \caption{
    Common ways to normalize a matrix of features using centering and scaling
    factors \(c_j\) and \(s_j\), respectively. Note that \(\bar{x}_j =
    n^{-1}\sum_{i=1}^n x_{ij}\).
  }
  \label{tab:normalization-types}
  \vskip 0.15in
  \small
  \begin{tabular}{lll}
    \toprule
    Normalization            & \(c_{j}\)          & \(s_j\)                                                   \\
    \midrule
    Standardization          & \(\bar{x}_j\)      & \(\sqrt{\frac{1}{n}\sum_{i=1}^n (x_{ij} - \bar{x}_j)^2}\) \\
    \addlinespace
    Max--Abs                 & 0                  & \(\max_i|x_{ij}|\)                                        \\
    \addlinespace
    Min--Max                 & \(\min_i(x_{ij})\) & \(\max_i(x_{ij}) - \min_i(x_{ij})\)                       \\
    \addlinespace
    \(\ell_1\)-Normalization & 0 or \(\bar{x}_j\) & \(\lVert \vec{x}_j\rVert_p\), \(p \in \{1,2,\dots\}\)     \\
    \addlinespace
    Adaptive Lasso           & 0                  & \(\beta_j^\text{OLS}\)                                    \\
    \bottomrule
  \end{tabular}
\end{table}

When the data is sparse, a common alternative to standardization is to scale features by
their maximum absolute values (max--abs normalization). This method has no impact on binary
data\footnote{Except in the extreme case when all values are 0.} and therefore retains
sparsity. For other types of data, it scales the features to take values in the range
\([-1, 1]\). Since the scaling is determined by a single value for each feature, the method
is naturally sensitive to outliers. For many types of data, such as normally distributed
data, it is also often the case that the sample maximum depends on sample size, which often
makes use of the method problematic~(\Cref{thm:maxabs-gev}~(\Cref{sec:additional-theory})

% TODO: should we study l1-normalization more?
Min-max normalization scales the data to lie in \([0, 1]\). As with the max--abs method,
min-max normalization retains sparsity and also shares its sensitivity to outliers and
sample size. Unlike max--abs scaling, however, min--max scaling is not sensitive to the
\emph{location} of the data, only its \emph{spread}. In \(\ell_1\)-normalization, each
feature is scaled with its sum of absolute value. The method is common in signal
processing. A special case of normalization is the adaptive lasso~\citep{zou2006}, which is
a two-step procedure. First we fit a model such as ordinary least-squares (OLS) or ridge
regression. Then we use the estimated coefficients to scale the features and refit.

\subsection{The Lasso, Ridge Regression, and the Elastic Net}

Let \((\hat{\beta}_0^{(n)}, \hat{\vec{\beta}}^{(n)})\) be a solution to the problem in
\Cref{eq:elastic-net}. Expanding \(f\) in \Cref{eq:elastic-net}, we have
\[
  \begin{aligned}
     & \frac{1}{2}\left( \vec y^\T \vec y - 2(\tilde{\mat{X}}\vec{\beta} + \beta_0)^\T\vec{y} + (\tilde{\mat{X}}\vec{\beta} + \beta_0)^\T(\tilde{\mat{X}}\vec{\beta} + \beta_0)\right) \\
     & + \lambda_1 \lVert \vec\beta \rVert_1 + \frac{\lambda_2}{2}\lVert \vec \beta \rVert_2^2.
  \end{aligned}
\]
Taking the subdifferential with respect to \(\vec{\beta}\) and \(\beta_0\), the KKT
stationarity condition yields the following system of equations:
\begin{equation}
  \label{eq:kkt-elasticnet}
  \begin{cases}
    \tilde{\mat{X}}^\T(\tilde{\mat{X}}\vec{\beta} + \beta_0 - \vec{y}) + \lambda_1 g + \lambda_2 \vec\beta \ni \vec{0}, \\
    n \beta_0 + (\tilde{\mat{X}}\vec{\beta})^\T \vec{1} - \vec{y}^\T \vec{1} = 0,
  \end{cases}
\end{equation}
where \(g\) is a subgradient of the \(\ell_1\) norm that has elements \(g_i\) such that
\[
  g_i \in
  \begin{cases}
    \{\sign{\beta_i}\} & \text{if } \beta_i \neq 0, \\
    [-1, 1]            & \text{otherwise}.
  \end{cases}
\]

\subsection{Weighted Penalization}%
\label{sec:weighted-elasticnet}

One alternative to normalization, which will be of particular interest in the case of the
elastic net, is to weight the coefficients in the penalty term---and instead of minimizing
\Cref{eq:elastic-net} minimize the following objective:
\[
  \begin{multlined}
    \tilde{f}(\beta_0, \vec{\beta}, \mat{X}, \vec{y}, \lambda_1, \lambda_2) =\\
    \frac{1}{2} \lVert \vec{y} - \beta_0 - \mat{X}\vec{\beta}\rVert_2^2 + \lambda_1 \sum_{j=1}^p u_j |\beta_j| + \frac{\lambda_2}{2} \sum_{j=1}^p v_j \beta_j^2,
  \end{multlined}
\]
with \(\vec{u}\) and \(\vec{v}\) being \(p\)-length vectors of positive scaling factors. We
call this the \emph{weighted elastic net}, which is equivalent to the original form when
\(u_j = s_j\) and \(v_j = s_j^2\), which can be seen by substituting \(\beta_js_j =
\tilde{\beta}_j\) in \Cref{eq:elastic-net} and solving for \(\tilde{\vec{\beta}}\).

\subsection{Orthogonal Features}

If the features of the normalized design matrix are orthogonal, that is,
\(\tilde{\mat{X}}^\intercal \tilde{\mat{X}} = \diag\left(\tilde{\vec{x}}_1^\T
\tilde{\vec{x}}_1, \dots, \tilde{\vec{x}}_p^\intercal \tilde{\vec{x}}_p\right) \), then
\Cref{eq:kkt-elasticnet} can be decomposed into a set of \(p + 1\) conditions:
%
\[
  \begin{cases}
    \tilde{\vec{x}}_j^\T (\tilde{\vec{x}}_j \beta_j + \ones \beta_0 - \vec{y}) + \lambda_2 \beta_j + \lambda_1 g \ni 0, & j \in [p], \\
    n \beta_0 + (\tilde{\mat{X}}\vec{\beta})^\T \vec{1} -  \vec{y}^\T \ones = 0.
  \end{cases}
\]
%
The inclusion of the intercept ensures that the locations (means) of the features do not
affect the solution (except for the intercept itself). We will therefore from now on assume
that the features are mean-centered so that \(c_j = \bar{\vec{x}}_j\) for all \(j\) and
therefore \(\tilde{\vec{x}}_j^\T \ones = 0\). A solution to the system of equations is then
given by the following set of equations~\citep{donoho1994}:
%
\begin{equation*}
  \label{eq:orthogonal-solution-normalized}
  \hat{\beta}^{(n)}_j = \frac{\st_{\lambda_1}\left(\tilde{\vec{x}}_j^\T \vec{y}\right)}{\tilde{\vec{x}}_j^\T \tilde{\vec{x}}_j + \lambda_2},
  \qquad
  \hat{\beta}_0^{(n)} = \frac{\vec{y}^\T \ones}{n},
\end{equation*}
%
where \(\st_\lambda(z) = \sign(z) \max(|z| - \lambda, 0)\) is the soft-thresholding
operator.

\subsection{Rescaling Regression Coefficients}

Normalization changes the optimization problem and its solution, the coefficients, which
will now be on the scale of the normalized features. But we are interested in
\(\hat{\vec{\beta}}\): the coefficients on the scale of the original problem. To obtain
these, we transform the coefficients from the normalized poblem, \(\hat\beta^{(n)}_j\),
back via
%
\begin{equation}
  \label{eq:orthogonal-solution}
  \hat\beta_j = \frac{\hat\beta^{(n)}_j}{s_j} \quad\text{for}\quad j = 1,2,\dots,p.
\end{equation}
%
There is a similar transformation for the intercept but we omit here since we are not
interested in it. Note that this rescaling is not needed in the weighted elastic
net~(\Cref{sec:weighted-elasticnet}), since the estimated coefficients are kept on the same
scale as the original (non-normalized) data.
