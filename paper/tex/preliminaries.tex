\section{Normalization}
To avoid possible confusion regarding the ambiguous use of terminology in the literature,
we will begin by clarifying what we mean by \emph{normalization}, which we define as the
process of centering and scaling the feature matrix.

\begin{definition}[Normalization]
  \label{def:normalization}
  Let \(\bm{X} \in \mathbb{R}^{n\times p}\) be the feature matrix and let
  \(\vec{c} \in \mathbb{R}^p\) and \(\vec{s} \in \mathbb{R}^p_+\) be centering
  and scaling factors respectively. Then \(\tilde{\bm{X}}\) is the
  \emph{normalized} feature matrix with elements given by
  \(\tilde{x}_{ij} = (x_{ij} - c_j)/s_j\).
\end{definition}

Some authors refer to the procedure in \Cref{def:normalization} as \emph{standardization},
but here we define standardization only as the case when centering with the mean and
scaling with the (uncorrected) standard deviation.

There are many different normalization strategies and we have listed a few common choices
in \Cref{tab:normalization-types}. Standardization is perhaps the most popular type of
normalization, at least in the field of statistics. One of its benefits is that it
simplifies certain aspects of fitting the model, such as fitting the intercept. The
downside of standardization is that it involves centering by the mean, which destroys
sparsity since centering shifts zero values to non-zero.

\begin{table}[hbtp]
  \centering
  \caption{
    Common ways to normalize a matrix of features using centering and scaling
    factors \(c_j\) and \(s_j\), respectively. Note that \(\bar{x}_j\) is
    the arithmetic mean of feature \(j\).
  }
  \label{tab:normalization-types}
  \begin{tabular}{lll}
    \toprule
    Normalization            & \(c_{j}\)          & \(s_j\)                                                   \\
    \midrule
    Standardization          & \(\bar{x}_j\)      & \(\sqrt{\frac{1}{n}\sum_{i=1}^n (x_{ij} - \bar{x}_j)^2}\) \\
    \addlinespace
    Max--Abs                 & 0                  & \(\max_i|x_{ij}|\)                                        \\
    \addlinespace
    Min--Max                 & \(\min_i(x_{ij})\) & \(\max_i(x_{ij}) - \min_i(x_{ij})\)                       \\
    \addlinespace
    \(\ell_1\)-Normalization & 0 or \(\bar{x}_j\) & \(\lVert \vec{x}_j\rVert_1\)                              \\
    \bottomrule
  \end{tabular}
\end{table}

When \(\bm{X}\) is sparse, two common alternatives to standardization are min--max and
max--abs (maximum absolute value) normalization, which scale the data to lie in \([0, 1]\)
and \([-1, 1]\) respectively, and therefore retain sparsity when features are binary. These
methods are, however, both sensitive to outliers. And since sample extreme values often
depend on sample size, as in the case of normal data~(\Cref{sec:maxabs-theory}), use of
these methods may sometimes be problematic. In the next section, we will examine how the
choice of normalization affects the estimates for the lasso, ridge, and elastic net
regression.

