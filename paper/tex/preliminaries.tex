\section{Normalization}

To avoid possible confusion regarding the ambiguous use of terminology in the literature,
we will begin by clarifying what we mean by \emph{normalization}, which we define as the
process of centering and scaling the feature matrix.

\begin{definition}[Normalization]
  \label{def:normalization}
  Let \(\bm{X} \in \mathbb{R}^{n\times p}\) be the feature matrix and let
  \(\vec{c} \in \mathbb{R}^p\) and \(\vec{s} \in \mathbb{R}^p_+\) be centering
  and scaling factors respectively. Then \(\tilde{\bm{X}}\) is the
  \emph{normalized} feature matrix with elements given by
  \(\tilde{x}_{ij} = (x_{ij} - c_j)/s_j\), where $i$ and $j$ index rows and
  columns, respectively.
\end{definition}

Some authors refer to the procedure in \Cref{def:normalization} as \emph{standardization},
but here we define standardization only as the case when centering with the mean and
scaling with the (uncorrected\footnote{Standard deviation computed without Bessel's
  correction (use of \(n-1\) instead of \(n\) in the standard deviation formula).}) standard
deviation.

There are many different normalization strategies and we have listed a few common choices
in \Cref{tab:normalization-types}. Standardization is perhaps the most popular type of
normalization, at least in the field of statistics. One of its benefits is that it
simplifies certain aspects of fitting the model, such as fitting the intercept. The
downside of standardization is that it involves centering by the mean, which destroys
sparsity in \(\bm{X}\) since centering shifts zero values to non-zero.

\begin{table}[t]
  \centering
  \caption{
    Common ways to normalize a matrix of features using centering and scaling
    factors \(c_j\) and \(s_j\), respectively. Note that \(\bar{x}_j\) is
    the arithmetic mean of feature \(j\) and that \(Q_a(\bm{x}_j)\) is the
    \(a\)th quartile of feature \(j\).
  }
  \label{tab:normalization-types}
  \begin{tabular}{lll}
    \toprule
    Normalization            & \(c_{j}\)          & \(s_j\)                                                     \\
    \midrule
    Standardization          & \(\bar{x}_j\)      & \(\frac{1}{\sqrt{n}} \lVert \vec{x}_j - \bar{x}_j\rVert_2\) \\
    \addlinespace
    \(\ell_1\)-Normalization & \(\bar{x}_j\)      & \(\frac{1}{\sqrt{n}} \lVert \vec{x}_j - \bar{x}_j\rVert_1\) \\
    \addlinespace
    Max--Abs                 & 0                  & \(\max_i|x_{ij}|\)                                          \\
    \addlinespace
    Min--Max                 & \(\min_i(x_{ij})\) & \(\max_i(x_{ij}) - \min_i(x_{ij})\)                         \\
    \addlinespace
    Robust Normalization     & \(Q_2(\bm{x}_j)\)  & \(Q_3(\bm{x}_j) - Q_1(\bm{x}_j)\)                           \\
    \addlinespace
    Adaptive Lasso           & 0                  & \(|\hat{\beta}_j^\text{OLS}|\)                              \\
    \bottomrule
  \end{tabular}
\end{table}

When \(\bm{X}\) is sparse, two common alternatives to standardization are min--max and
max--abs (maximum absolute value) normalization, which scale the data to lie in \([0, 1]\)
and \([-1, 1]\) respectively, and therefore retain sparsity when features are binary. These
methods are, however, both sensitive to outliers. And since sample extreme values often
depend on sample size, as in the case of normal data~(\Cref{sec:maxabs-theory}), use of
these methods may sometimes be problematic. Another alternative is to replace the
\(\ell_2\)-norm in the standardization method with the \(\ell_1\)-norm, which leads to
\(\ell_1\)-normalization. We note here that this method is often used without centering,
but this would make the method depend on the mean of the feature, as in the case of
max--abs normalization, and we therefore prefer the centered version here.

We have also included robust normalization in \Cref{tab:normalization-types}, which is a
version of normalization that uses the median and interquartile range (IQR) as centering
and scaling factors. Finally, we have also included the adaptive lasso~\citep{zou2006a},
which is a special case of normalization that fits a standard ordinary least-squares
regression (OLS) model to the data and uses the OLS estimates as scaling
factors.\footnote{In the case when \(p \gg n\), a ridge estimator is typically used
  instead.} We have both of these methods in \Cref{tab:normalization-types} for completeness,
but we will not study it further in this paper.

In the next section, we will examine how the choice of normalization affects the estimates
for the lasso, ridge, and elastic net regression.

