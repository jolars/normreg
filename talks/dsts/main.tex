\PassOptionsToPackage{unicode,pdfusetitle}{hyperref}
\PassOptionsToPackage{hyphens}{url}
\PassOptionsToPackage{dvipsnames,svgnames,x11names}{xcolor}

\documentclass[aspectratio=1610,onlytextwidth]{beamer}

\usetheme{moloch}
% \molochset{block=fill}
\usefonttheme{professionalfonts}
\setbeamertemplate{page number in head/foot}[appendixframenumber]

\usepackage{lmodern}
\usepackage{amssymb,amsmath,mathtools,amsthm}
\usepackage[T1]{fontenc}
\usepackage{textcomp}

\usepackage{upquote} % straight quotes in verbatim environments
\usepackage{microtype}
\UseMicrotypeSet[protrusion]{basicmath} % disable protrusion for tt fonts

\usepackage{xcolor}
\usepackage{xurl} % add URL line breaks if available
\usepackage{bookmark}
\usepackage{hyperref}
\usepackage{bm}

\hypersetup{%
  colorlinks = true,
  linkcolor  = mLightGreen,
  filecolor  = mLightGreen,
  citecolor  = mLightGreen,
  urlcolor   = mLightGreen
}

%% subfigures
\usepackage{caption}
\usepackage{subcaption}

% % algorithms
% \usepackage[ruled,vlined]{algorithm2e}
% \resetcounteronoverlays{algocf}

\usepackage{booktabs}

% bibliography
\usepackage[style=authoryear]{biblatex}
\addbibresource{bibliography.bib}

% title block
\titlegraphic{\hfill\includegraphics[width=1.8cm]{figures/logo.pdf}}
\title{The Choice of Normalization Directly Affects Feature Selection in Regularized Regression}
\subtitle{DSTS's Two-Day Meeting, Autumn 2024}
\author{Johan Larsson\\\smallskip\scriptsize \url{https://jolars.co}, {\texttt{@jolars@mastodon.social}}}
\institute{Department of Mathematical Sciences, Copenhagen University}
\date{November 13, 2024}

% operators
\DeclareMathOperator*{\argmax}{arg\,max}
\DeclareMathOperator*{\argmin}{arg\,min}
\DeclareMathOperator{\E}{E}
\DeclareMathOperator{\var}{Var}
\DeclareMathOperator{\cov}{Cov}
\DeclareMathOperator{\tr}{tr}
\DeclareMathOperator{\diag}{diag}
\DeclareMathOperator{\range}{range}
\DeclareMathOperator{\nullspace}{null}
\DeclareMathOperator{\rank}{rank}
\DeclareMathOperator{\card}{card}
\DeclareMathOperator{\sign}{sign}
\DeclareMathOperator{\st}{S}
\DeclareMathOperator{\normal}{Normal}
\DeclareMathOperator{\fnormal}{FoldedNormal}
\DeclareMathOperator{\bernoulli}{Bernoulli}
\DeclareMathOperator{\erf}{erf}
\DeclareMathOperator{\mse}{MSE}
\DeclareMathOperator{\risk}{R}
% \DeclareMathOperator{\I}{I}
% \DeclareMathOperator{\T}{}

\renewcommand{\vec}{\symbf}
\newcommand{\mat}{\symbf}
\newcommand*\du{\mathop{}\!\mathrm{d}}
\newcommand{\T}{\intercal}
\newcommand{\ones}{\symbf{1}}
\newcommand{\ind}[1]{\operatorname{I}_{#1}}

% \newcommand{\todojl}[1]{\todo[color=green!40]{#1}}

% \newcommand{\mv}[1]{{\boldsymbol{\mathrm{#1}}}}



\begin{document}

\maketitle

\begin{frame}[c]
  \frametitle{This Talk}

  \begin{alertblock}{Problem and Motivation}
    Feature normalization has large effects in regularized regression (lasso, ridge)
    but there is no research on this.
  \end{alertblock}

  \pause

  \begin{columns}[T]
    \begin{column}{0.55\textwidth}
      \begin{exampleblock}{Results}
        \begin{itemize}
          \item Class balance has a normalization-dependent impact on the model.
          \item In mixed data, choice of normalization implictly weighs features' importances.
        \end{itemize}
      \end{exampleblock}

    \end{column}
    \begin{column}{0.4\textwidth}
      \begin{figure}
        \includegraphics[width=\textwidth]{figures/jonas.jpg}
        \caption{Joint work with Jonas Wallin}
      \end{figure}
    \end{column}
  \end{columns}

\end{frame}

\begin{frame}
  \frametitle{Overview}

  \tableofcontents
\end{frame}

\section{Preliminaries}

\begin{frame}
  \frametitle{General Setup}

  \begin{itemize}
    \item Data consists of a \alert{fixed} matrix of features \(\mat{X} \in \mathbb{R}^{n \times p}\)
          and a response vector \(\vec{y} \in \mathbb{R}^n\).
    \item \(\vec{y}\) comes from a linear model, that is,
          \[
            y_i = \beta_0^* + \vec x_i^\T \vec\beta^* + \varepsilon_i \quad\text{for} \quad i \in 1,\dots,n,
          \]
          where \(\vec{\beta}^*\) is the vector of \emph{true} coefficients.
    \item \(\varepsilon_i\) is the measurement noise, generated from some random variable.
  \end{itemize}
\end{frame}

\begin{frame}[c]
  \frametitle{The Elastic Net}

  Linear regression plus a combination of the \(\ell_1\) and \(\ell_2\) penalties:
  \begin{equation*}
    \vec{\beta}^* = \operatorname*{arg\,min}_{\beta \in \mathbb{R}^p} \bigg( \frac{1}{2} \lVert \vec y - \mat{X}\vec{\beta} \rVert^2_2  + \underbrace{\lambda_1 \lVert \vec\beta \rVert_1}_\text{lasso} + \underbrace{\frac{\lambda_2}{2}\lVert \vec \beta \rVert_2^2}_\text{ridge}\bigg)
  \end{equation*}

  \pause

  \begin{figure}
    \centering
    \includegraphics[]{figures/elasticnet-balls.pdf}
    \caption{%
      Elastic net is a combination of the lasso and ridge.
    }
  \end{figure}
\end{frame}

\begin{frame}[c]
  \frametitle{Regularization}

  \begin{figure}[htpb]
    \centering
    \includegraphics[]{figures/regularization-illustration.pdf}
    \caption{%
      Regularization for a simple linear regression problem
    }
  \end{figure}

  \pause
  \begin{columns}[T]
    \begin{column}{0.45\textwidth}
      \begin{block}{Why Regularize? (Lasso, Ridge)}
        \begin{itemize}
          \item Uniqueness when \(p \gg n\)
          \item To overcome overfitting
        \end{itemize}
      \end{block}
    \end{column}
    \pause
    \begin{column}{0.45\textwidth}
      \begin{block}{Why Sparsity? (Lasso)}
        \begin{itemize}
          \item Interpretability
          \item The sparsity bet
        \end{itemize}
      \end{block}
    \end{column}
  \end{columns}

\end{frame}

% TODO: add a simple demonstration using linear regression of this

\begin{frame}[c]
  \frametitle{The Elastic Net Path}

  \begin{columns}
    \begin{column}{0.55\textwidth}
      \begin{itemize}[<+->]
        \item Don't know optimal \(\lambda_1\) and \(\lambda_2\) in advance.
        \item Instead we use hyper-parameter optimization (e.g. cross-validation).
        \item Common parametrization:
              \begin{align*}
                \lambda_1 & = \alpha\lambda,     \\
                \lambda_2 & = (1- \alpha)\lambda
              \end{align*}
              with \(\alpha \in [0, 1]\).
        \item For each \(\alpha\), solve the elastic net over a sequence of \(\lambda\): the
              \textbf{elastic net path}.
      \end{itemize}
    \end{column}
    \begin{column}{0.4\textwidth}
      \only<4->{%
        \begin{figure}[htpb]
          \centering
          \includegraphics[]{figures/elasticnet_path.pdf}
          \caption{%
            The elastic net path
          }
        \end{figure}
      }
    \end{column}
  \end{columns}

\end{frame}

\begin{frame}[c]
  \frametitle{Sensitivity to Scale}

  Lasso and ridge penalties are \textbf{norms}---feature scales matter!

  \pause

  \begin{exampleblock}{Example}
    Assume
    \[
      \mat{X} \sim \operatorname{Normal}\left(\begin{bmatrix}0 \\ 0\end{bmatrix}, \begin{bmatrix} 4 & 0 \\ 0 & 1\end{bmatrix}\right), \qquad \vec{\beta}^* = \begin{bmatrix} \frac{1}{2} \\ 1 \end{bmatrix}.
    \]

    \medskip\pause

    \begin{table}
      \begin{tabular}{lcc}
        \toprule
        Model & \(\hat{\bm{\beta}}\)                                   & \(\hat{\vec{\beta}}_\text{std}\)                             \\
        \midrule
        OLS   & \(\begin{bmatrix} 0.50 & 1.00\end{bmatrix}^\intercal\) & \(\begin{bmatrix}1.00 & 1.00\end{bmatrix}^\intercal\) \pause \\
        Lasso & \(\begin{bmatrix} 0.38 & 0.50\end{bmatrix}^\intercal\) & \(\begin{bmatrix}0.74 & 0.50\end{bmatrix}^\intercal\) \pause \\
        Ridge & \(\begin{bmatrix} 0.37 & 0.41\end{bmatrix}^\intercal\) & \(\begin{bmatrix}0.74 & 0.41\end{bmatrix}^\intercal\)        \\
        \bottomrule
      \end{tabular}
    \end{table}
  \end{exampleblock}

  \pause

  \alert{Large} scale means \alert{less} penalization because the size of \(\beta_j\) can be smaller for an equivalent effect (on \(\vec{y}\)).

\end{frame}

\begin{frame}[c]
  \frametitle{Normalization}

  \begin{itemize}[<+->]
    \item Scale sensitivity can be mitigated by normalizing the features.
    \item Let \(\tilde{\mat X}\) be the normalized feature matrix, with elements
          \[
            \tilde{x}_{ij} = \frac{x_{ij} - c_{j}}{s_j}.
          \]
    \item After fitting, we transform the coefficients back to their original scale via
          \[
            \hat\beta_j = \frac{\hat\beta^{(n)}_j}{s_j} \quad\text{for}\quad j = 1,2,\dots,p,
          \]
          where \(\hat\beta^{(n)}_j\) is a coefficient from the normalized problem.
  \end{itemize}

  % \bigskip\pause
  % \begin{block}{Ambiguous Nomenclature}
  %   \begin{itemize}
  %     \item Some refer to this as \emph{standardization} (which we dedicate for the mean--standard deviation combo).
  %     \item Some take normalization to mean \alert{sample-wise} normalization.
  %     \item Some take normalization to mean scaling with a \alert{norm}.
  %     \item Some refer to this (centering and scaling) as \alert{scaling}.
  %   \end{itemize}
  % \end{block}
\end{frame}

\begin{frame}[c]
  \begin{table}[hbt]
    \centering
    \caption{Common ways to normalize \(\mat{X}\)}
    \begin{tabular}{lll}
      \toprule
      Normalization    & Centering (\(c_{j}\))                          & Scaling (\(s_j\))                                         \\
      \midrule
      Standardization  & \(\bar{x}_j = \frac{1}{n}\sum_{i=1}^n x_{ij}\) & \(\sqrt{\frac{1}{n}\sum_{i=1}^n (x_{ij} - \bar{x}_j)^2}\) \\
      \addlinespace
      Min--Max         & \(\min_i(x_{ij})\)                             & \(\max_i(x_{ij}) - \min_i(x_{ij})\)                       \\
      \addlinespace
      Unit Vector (L2) & 0                                              & \(\sqrt{\sum_{i=1}^n x_{ij}^2}\)                          \\
      \addlinespace
      Max--Abs         & 0                                              & \(\max_i(|x_{ij}|)\)                                      \\
      \addlinespace
      Adaptive Lasso   & 0                                              & \(\beta_j^\text{OLS}\)                                    \\
      \bottomrule
    \end{tabular}
  \end{table}
\end{frame}

\section{Motivation and Aims}

\begin{frame}[c]
  % \frametitle{The Type of Normalization Matters}

  \begin{figure}[htpb]
    \centering
    \includegraphics[width=0.9\textwidth]{figures/paper6-realdata_paths.pdf}
    \caption{%
      Normalization matters. Lasso paths under two different types of normalization (standardization and max--abs normalization). The union of the first ten features selected in any of the settings are colored.
    }
  \end{figure}

\end{frame}

\begin{frame}[c]
  \frametitle{Motivation}

  \begin{block}{Motivation}
    \begin{itemize}[<+->]
      \item So, normalization matters but there has been no research into this.
      \item Everyone agrees you need to normalize, but how to do so is usually motivated by being
            ``standard''.
      \item Documentation for popular machine learning packages advocate different normalization
            strategies when data is sparse.
      \item Approach depends on field: statisticians standardize, signal processors use $\ell_1$
            normalization, machine learning people scale to $[0, 1]$ or $[-1, 1]$.)
    \end{itemize}
  \end{block}

  \pause

  \begin{block}{Aims}
    \begin{itemize}[<+->]
      \item Binary features, particularly with respect to the \alert{class balance} thereof
      \item A mix of binary and normally distributed features
    \end{itemize}
  \end{block}

\end{frame}

\section{Results}

\begin{frame}[c]
  \frametitle{Orthogonal Features}

  There is no explicit solution to the elastic net problem in general (unless \(\lambda_1 =
  0\)).

  \pause\bigskip

  \begin{columns}
    \begin{column}{0.5\textwidth}
      But if we assume that the features are orthogonal, that is
      \[
        \tilde{\mat{X}}^\intercal \tilde{\mat{X}} = \diag(\tilde{\vec{x}}_1^\intercal \tilde{\vec{x}}_1, \dots, \tilde{\vec{x}}_p^\intercal \tilde{\vec{x}}_p),
      \]
      then there is:\footnote[frame]{We have also assumed that the features are mean-centered
        here.}:
      \begin{equation*}
        \hat\beta_j = \frac{\st_{\lambda_1}\left(\tilde{\vec{x}}_j^\T \vec{y}\right)}{s_j\left(\tilde{\vec{x}}_j^\T \tilde{\vec{x}}_j + \lambda_2\right)},
      \end{equation*}
      where
      \[
        \st_\lambda(z) = \sign(z) \max(|z| - \lambda, 0).
      \]\pause
    \end{column}
    \begin{column}{0.4\textwidth}
      \begin{figure}[htpb]
        \centering
        \includegraphics[]{figures/soft-thresholding.pdf}
        \caption{%
          Soft thresholding
        }
      \end{figure}
    \end{column}
  \end{columns}

\end{frame}

\begin{frame}[c]
  \frametitle{Bias and Variance of the Elastic Net Estimator}

  The goal is computing the expected value of the elastic net estimator,
  \[
    \E \hat\beta_j = \frac{\E \st_{\lambda_1}\left(\tilde{\vec{x}}_j^\T \vec{y}\right)}{s_j\left(\tilde{\vec{x}}_j^\T \tilde{\vec{x}}_j + \lambda_2\right)},
  \]
  since we treat \(\mat{X}\) as fixed.

  \bigskip\pause

  Letting \(Z = \tilde{\vec{x}}^\intercal \vec{y}\) and assuming that \(\varepsilon_i\) is
  i.i.d. Normally-distributed with mean zero and finite variance \(\sigma_\varepsilon^2\), we
  have
  \[
    Z \sim \normal\left(\mu = \tilde{\vec{x}}_j^\T\vec{x}_j \beta_j, \sigma^2 =  \sigma_\varepsilon^2 \lVert \tilde{\vec{x}}_j\rVert_2^2 \right).
  \]

  % \begin{align*}
  %   % \E Z   & = \mu = \E \left( \tilde{\vec{x}}_j^\T (\vec{x}_j\beta_j + \vec{\varepsilon}) \right)  = \tilde{\vec{x}}_j^\T\vec{x}_j \beta_j,  \\
  %   % \var Z & = \sigma^2 = \var\left(\tilde{\vec{x}}_j ^\T \vec{\varepsilon}\right) = \sigma_\varepsilon^2 \lVert \tilde{\vec{x}}_j\rVert_2^2.
  %   \E Z   & = \mu = \tilde{\vec{x}}_j^\T\vec{x}_j \beta_j,                        \\
  %   \var Z & = \sigma^2 = \sigma_\varepsilon^2 \lVert \tilde{\vec{x}}_j\rVert_2^2.
  % \end{align*}

  \bigskip

  Next, will will turn to \(\E \st_{\lambda_1}(Z)\).
\end{frame}

\begin{frame}[c]
  \frametitle{Bias}

  \begin{columns}[T]
    \begin{column}{0.5\textwidth}
      The expected value of the soft-thresholding estimator is
      \begin{align*}
        \E \st_\lambda(Z) & = \int_{-\infty}^\infty \st_\lambda(z) f_Z(z) \du z                                                   \nonumber \\
        % & = \int_{-\infty}^\infty \ind{|z| > \lambda} (z -\sign(z)\lambda) f_Z(z) \du z                         \nonumber \\
                          & = \int_{-\infty}^{-\lambda}(z + \lambda)f_Z(z) \du z                                                            \\
                          & \phantom{={}} + \int_{\lambda}^\infty (z - \lambda)f_Z(z) \du z.
      \end{align*}

      \only<2->{
        The bias of \(\hat\beta_j\) is
        \begin{equation*}
          \E \hat\beta_j - \beta_j^* = \frac{1}{d_j}\E \st_\lambda(Z) - \beta^*_j,
        \end{equation*}
        where \(d_j = s_j \left(\tilde{\boldsymbol{x}_j}^\intercal\tilde{\boldsymbol{x}_j} + \lambda_2\right)\).
      }
    \end{column}
    \begin{column}{0.4\textwidth}
      \begin{figure}[htpb]
        \centering
        \includegraphics[]{figures/z_normal.pdf}
        \caption{%
          Distributions of \(Z\) and its value after soft-thresholding.
        }
      \end{figure}
    \end{column}
  \end{columns}

\end{frame}

\begin{frame}[c]
  \frametitle{Variance}

  The variance of the soft-thresholding estimator is
  \begin{equation*}
    \var {S_\lambda(Z)} = \int_{-\infty}^{-\lambda}(z + \lambda)^2f_Z(z) \du z + \int_{\lambda}^\infty (z - \lambda)^2 f_Z(z) \du z - \left(\E \st_\lambda(Z)\right)^2
  \end{equation*}
  and consequently the variance of the elastic net estimator is
  \begin{equation*}
    \var \hat\beta_j = \frac{1}{d_j^2} \var \st_\lambda(Z).
  \end{equation*}

\end{frame}

% \begin{frame}[c]
%   \frametitle{Normally Distributed Noise}
%
%   We now assume that \(\varepsilon_i \sim \normal{(0, \sigma_\varepsilon^2)}\), which means
%   that
%   \[
%     Z \sim \normal\left(\mu = \tilde{\vec{x}}_j^\T\vec{x}_j \beta_j, \sigma^2 =  \sigma_\varepsilon^2 \lVert \tilde{\vec{x}}_j\rVert_2^2 \right).
%   \]
%
%   Let \(\theta = -\mu -\lambda_1 \) and \(\gamma = \mu - \lambda_1\). Then the expected value
%   of soft-thresholding of \(Z\) is
%   \begin{align*}
%     \E \st_{\lambda_1}(Z) %& = \int_{-\infty}^\frac{\theta}{\sigma} (\sigma u - \theta) \pdf(u) \du u + \int_{-\frac{\gamma}{\sigma}}^\infty (\sigma u + \gamma) \pdf(u) \du u                                               \nonumber \\
%      & = -\theta \cdf\left(\frac{\theta}{\sigma}\right) - \sigma \pdf\left(\frac{\theta}{\sigma}\right) + \gamma \cdf\left(\frac{\gamma}{\sigma}\right) + \sigma \pdf\left(\frac{\gamma}{\sigma}\right)
%   \end{align*}
%   where \(\pdf(u)\) and \(\cdf(u)\) are the pdf and cdf of the standard normal distribution, respectively.
%
%   \bigskip\pause
%
%   We have an expression for \(\var \st_{\lambda_1}(Z)\) too (but omit it here).
%
% \end{frame}

\begin{frame}[c]
  \frametitle{Binary Features}
  Recall that
  \[
    Z \sim \normal\left(\mu = \tilde{\vec{x}}_j^\T\vec{x}_j \beta_j, \sigma^2 =  \sigma_\varepsilon^2 \lVert \tilde{\vec{x}}_j\rVert_2^2 \right)
  \]
  and assume we have a binary feature \(\vec{x}_j\), such that \(x_{ij} \in \{0, 1\}\). Let
  \(q \in [0, 1]\) be the class balance of this feature, that is: \(q = \bar{\vec{x}}_j\).

  \bigskip

  In this case, we observe that
  \[
    \begin{aligned}
      % \tilde{\vec{x}}_j^\T \tilde{\vec{x}}_j & = \frac{1}{s_j^2}(\vec{x}_j - \ones c_j)^\T (\vec{x}_j - \ones c_j) = \frac{1}{s^2_j}(nq - 2nq^2 + nq^2) = \frac{n(q-q^2)}{s^2_j}, \\
      % \tilde{\vec{x}}_j^\T \vec{x}_j         & = \frac{1}{s_j}(\vec{x}_j^\T \vec{x}_j - \vec{x}_j^\T \ones c_j) = \frac{n(q - q^2)}{s_j}.
      \lVert \tilde{\bm{x}}_j\rVert_2^2 & = \frac{n(q-q^2)}{s^2_j}, \\
      \tilde{\vec{x}}_j^\T \vec{x}_j    & = \frac{n(q - q^2)}{s_j}.
    \end{aligned}
  \]
  \pause%
  And consequently
  \[
    \mu = \frac{\beta^*_j n(q - q^2)}{s_j}, \qquad \sigma^2 = \frac{\sigma_\varepsilon^2n(q - q^2)}{s^2_j}, \qquad d_j = \frac{n(q -q^2)}{s_j}  + \lambda_2 s_j.
  \]
\end{frame}

\begin{frame}[c]
  \frametitle{Noiseless Case for Binary Features}

  In the noiseless case, we have
  \[
    \hat{\beta}_j = \frac{\st_{\lambda_1}(\tilde{\vec{x}}^\intercal \vec{y})}{s_j\left(\tilde{\vec{x}}_j^\intercal \tilde{\vec{x}}_j + \lambda_2\right)}
    =
    \frac{\operatorname{S}_{\lambda_1}\left(\frac{\beta_j^* n \alert{(q - q^2)}}{s_j}\right)}{s_j\left(\frac{n\alert{(q - q^2)}}{s_j^2} + \lambda_2\right)}.
  \]
  \pause
  \begin{itemize}[<+->]
    \item Means that the elastic net estimator depends on class balance (\(q\)).
    \item \(s_j = q - q^2\) for lasso and \(s_j = \sqrt{q-q^2}\) for ridge removes effect of \(q\).
    \item Suggests the parametrization
          \[
            s_j = (q - q^2)^\delta, \qquad \delta \geq 0,
          \]
          which we will rely on for the rest of the talk.
    \item Indicates there might be no (simple) \(s_j\) that will work for the elastic net.
  \end{itemize}
  % \pause
  % \begin{block}{Scaling Parametrization}
  %   We are going to use the scaling parametrization
  %   \[
  %     s_j = (q - q^2)^\delta, \qquad \delta \geq 0.
  %   \]
  % \end{block}
\end{frame}

\begin{frame}[c]
  \frametitle{Probability of Selection}

  Since \(\mat{X}\) is fixed and \(\vec{\varepsilon}\) is normal, we can compute the
  probability of selection:
  \[
    \Pr(\hat{\beta}_j \neq 0) = \cdf\left(\frac{\mu - \lambda_1}{\sigma}\right) + \cdf\left(\frac{- \mu -\lambda_1}{\sigma}\right).
  \]

  \begin{figure}[htpb]
    \centering
    \includegraphics[width=\textwidth]{figures/selection_probability.pdf}
    \caption{%
      Probability that the elastic net selects a feature across different noise levels \((\sigma_\varepsilon)\), types of normalization (\(\delta\)), and class balance (\(q\)).
      The dashed line is asymptotic behavior for \(\delta = 1/2\).
      Scaling used is \(s_j \propto (q - q^2)^\delta\).
    }
  \end{figure}
\end{frame}

\begin{frame}[c]
  \frametitle{Implications}
  \begin{columns}[T]
    \begin{column}{0.45\textwidth}
      \begin{block}{Rare Traits}
        Features with large class-imbalances might not be selected even if effect is \alert{very strong}
        (e.g. rare SNPs, mutations).
      \end{block}
    \end{column}
    \pause
    \begin{column}{0.45\textwidth}
      \begin{block}{Subgroup Data}
        Results become dependent on data colection. \medskip

        Collecting more data with different class balances influences the results (since class
        balances change).
      \end{block}

    \end{column}
  \end{columns}
\end{frame}

\begin{frame}[c]
  \frametitle{Asymptotic Results for Bias and Variance}

  \begin{theorem}
    If \(\vec{x}_j\) is a binary feature with class balance \(q \in (0, 1)\) and \(\lambda_1,\lambda_2 \in (0,\infty)\), \(\sigma_\varepsilon > 0\), and \(s_j = (q - q^2)^{\delta}\), \(\delta \geq 0\), then
    \[
      \lim_{q \rightarrow 1^-} \E \hat{\beta}_j =
      \begin{cases}
        0                                                                                                  & \text{if } 0 \leq \delta < \frac{1}{2}, \\
        \frac{2n \beta_j^*}{n + \lambda_2} \cdf\left(-\frac{\lambda_1}{\sigma_\varepsilon \sqrt{n}}\right) & \text{if } \delta = \frac{1}{2},        \\
        \beta^*_j                                                                                          & \text{if } \delta \geq \frac{1}{2}.
      \end{cases}
    \]
    \pause and
    \[
      \lim_{q \rightarrow 1^-} \var \hat{\beta}_j =
      \begin{cases}
        0      & \text{if } 0 \leq \delta < \frac{1}{2}, \\
        \infty & \text{if } \delta \geq \frac{1}{2}.
      \end{cases}
    \]
  \end{theorem}

\end{frame}

\begin{frame}[c]
  % \frametitle{A Bias--Variance Tradeoff}

  \begin{figure}
    \centering
    \includegraphics[width=0.83\textwidth]{figures/bias-var-onedim.pdf}
    \caption{%
      A bias variance tradeoff. Bias, variance, and mean-squared error for a one-dimensional lasso problem. Theoretical result for orthogonal features. Dotted line is asymptotic result or \(\delta = 1/2\).
      Scaling used is \(s_j \propto (q - q^2)^\delta\).
    }
  \end{figure}

\end{frame}

\begin{frame}[c]
  % \frametitle{Multiple Features: Power, FDR, and NMSE}

  % Lasso example with 10 true signals and varying \(q\) and \(p\).

  \begin{figure}[htpb]
    \centering
    \subcaptionbox{%
      Power in the sense of detecting all the true signals. Constant \(p\).
    }{\includegraphics[height=5cm]{figures/power.pdf}}\hfill%
    \only<2->{%
      \subcaptionbox{%
        False discovery rate (FDR) and normalized mean-squared error (NMSE).
      }{\includegraphics[height=5cm]{figures/fdr_mse.pdf}}
    }
    \caption{%
      Multiple features: 10 true signals and varying \(q\) and \(p\). Mean
      squared error (MSE), false discovery rate (FDR), and power
    }
  \end{figure}
\end{frame}

\begin{frame}[c]
  \frametitle{Mixed Data}
  \begin{columns}
    \begin{column}{0.45\textwidth}
      \textbf{So far:} \alert{all} binary features. What about mixing binary and continuous (normal) features?

      \medskip

      How to put binary features and normal features on the ``same'' scale?
    \end{column}
    \begin{column}{0.45\textwidth}
      \begin{figure}[htpb]
        \centering
        \includegraphics[]{figures/mixed-data-comp.pdf}
        \caption{%
          How do we match these?
        }
      \end{figure}
    \end{column}
  \end{columns}
\end{frame}

\begin{frame}[c]
  \frametitle{Mixed Data}

  \begin{block}{Our Definition of Comparability}
    The effects of a binary feature and a normally distributed feature are \alert{comparable} if a flip in the binary feature has the same effect as a two-standard deviation change in the normal feature~\citep{gelman2008}.
  \end{block}

  \pause

  \begin{exampleblock}{Examples}
    Assume entries in \(\vec{x}_1\) are binary and \(\vec{x}_2\) come from a random variable \(X_2\). The effects are comparable in the following cases:
    \begin{itemize}
      \item \(X_2 \sim \normal(\mu, 1/2)\), \(\beta_1^* = 1\), and \(\beta_2^* = 1\).
      \item \(X_2 \sim \normal(\mu, 2)\), \(\beta_1^* = 1\), and \(\beta_2^* = 0.25\).
    \end{itemize}
  \end{exampleblock}
\end{frame}

\begin{frame}[c]
  \frametitle{Choice of Scaling in Mixed Data}

  For the two-standard deviation notion of comparability to hold, we need to modify our
  scaling factor \(s_j\).

  \bigskip\pause

  As before, we assume that \(\vec{x}_1\) is binary and \(X_2 \sim \normal(\mu, 1/2)\),
  \(\beta_1^* = \beta_2^* = 1\) so that they have \emph{comparable} effects. Also assume we
  standardize \(\vec{x}_2\).

  \medskip

  We want \(\hat{\beta}_1 = \hat{\beta}_2\). That is,
  \[
    \underbrace{\frac{\st_{\lambda_1}\left(\frac{n (q - q^2)}{s_j}\right)}{s_1\left(\frac{n(q - q^2)}{s_1^2} + \lambda_2\right)}}_{\hat{\beta}_1}  = \underbrace{\frac{\st_{\lambda_1}\left(\frac{n}{2}\right)}{\frac{1}{2}\left(n + \lambda_2\right)}}_{\hat{\beta}_2}.
  \]

  \medskip\pause

  The choice \(s_1 = (2 (q - q^2))^\delta\) works when classes are balanced (\(q = 0.5\)).
  But no clear choice for the elastic net case.
\end{frame}

\section{Experiments}

\begin{frame}[c]
  \frametitle{Binary Features (Decreasing \(q\))}

  \begin{figure}[htpb]
    \centering
    % \includegraphics[width=0.86\textwidth]{figures/binary_decreasing.pdf}
    \includegraphics{figures/binary_decreasing.pdf}
    \caption{%
      Lasso estimates for first 30 coefficients. First 20 features are true signals with a geometrically decreasing class balance from 0.5 to 0.99. \(\rho\) is
      a measure of autocorrelation.
    }
  \end{figure}
\end{frame}

\begin{frame}[c]
  \frametitle{Binary Features (Signal-to-Noise Ratio)}

  \begin{figure}[htpb]
    \centering
    \includegraphics{figures/binary_data_sim.pdf}
    \caption{%
      Normalized mean-squared test set error (NMSE).
    }
  \end{figure}
\end{frame}

\begin{frame}[c]
  \frametitle{Mixed Data}

  \begin{figure}[htpb]
    \centering
    \includegraphics[]{figures/paper6-mixed_data.pdf}
    \caption{%
      Comparison between lasso and ridge estimators for features generated to resemble features from various distributions.}
  \end{figure}
\end{frame}

\begin{frame}[c]
  \frametitle{Hyperparameter Optimization}

  % \textbf{Idea:} The choice of \(\delta\) affects the model, so let's optimize over it.

  \begin{figure}[htpb]
    \centering
    \includegraphics{figures/hyperopt_surfaces.pdf}
    \caption{%
      Contour plots of hold-out (validation set) error across a grid of \(\delta\) and \(\lambda\) values for the
      lasso and ridge.
    }
  \end{figure}

\end{frame}

\begin{frame}[c]
  \frametitle{Summary}
  \begin{exampleblock}{Conclusions}
    \begin{itemize}
      \item Class balance plays a crucial role when using regularized regression on binary data.
      \item As far as we know the first paper to investigate the interplay between normalization and
            regularization
      \item New scaling approach to deal with class-imbalanced binary features
      \item Discussion and suggestions for dealing with mixed data
    \end{itemize}
  \end{exampleblock}

  \pause

  \begin{alertblock}{Limitations}
    \begin{itemize}
      \item So far only theoretical results for limited cases:
            \begin{itemize}
              \item Fixed data (\(\mat{X}\)), normal noise
              \item Orthogonal features
              \item Normal and binary features
            \end{itemize}
    \end{itemize}
  \end{alertblock}
\end{frame}

\begin{frame}[standout]
  Thank you!
\end{frame}

\appendix

\begin{frame}[allowframebreaks]{References}
  \printbibliography[heading=none]
\end{frame}

\section{Extras}

\begin{frame}[c]
  \frametitle{Max--Abs Scaling of Continuous Features}

  \begin{columns}
    \begin{column}{0.45\textwidth}

      \begin{itemize}
        \item Min--max normalization is sometimes used in continuous data
        \item Very sensitive to outliers
        \item But also depend on sample size!
        \item In other words, results in model validation with varying sample sizes can yield very
              strange results.
      \end{itemize}

      % \begin{theorem}
      %   Let \(X_1, X_2, \dots, X_n\) be a sample of normally distributed random variables, each with mean \(\mu\) and standard deviation \(\sigma\). Then
      %   \[
      %     \lim_{n \rightarrow \infty}\Pr\left(\max_{i \in [n]} |X_i| \leq x\right) = G(x),
      %   \]
      %   where \(G\) is the cumulative distribution function of a Gumbel distribution with
      %   parameters
      %   \[
      %     b_n = F_Y^{-1}(1 - 1/n)\quad \text{and} \quad a_n = \frac{1}{n f_Y(\mu_n)},
      %   \]
      %   where \(f_Y\) and \(F_Y^{-1}\) are the probability distribution function and quantile function, respectively, of a folded normal distribution with mean \(\mu\) and standard deviation \(\sigma\).
      % \end{theorem}
    \end{column}
    \begin{column}{0.45\textwidth}
      \begin{figure}[htpb]
        \centering
        \includegraphics[width=0.7\textwidth]{figures/maxabs_gev.pdf}
        \includegraphics[width=0.7\textwidth]{figures/maxabs_n.pdf}
        \caption{%
          Effects of maximum absolute value scaling.
        }
      \end{figure}
    \end{column}
  \end{columns}

\end{frame}

\begin{frame}[c]
  \frametitle{Hyperparameter Optimization (Support and NMSE)}

  \begin{figure}[htpb]
    \centering
    \includegraphics[width=\textwidth]{figures/hyperopt_paths.pdf}
    \caption{%
      Support and NMSE of the lasso for different values of \(\delta\) and \(\lambda\).
    }
  \end{figure}
\end{frame}

\begin{frame}[c]
  \frametitle{Background on the Elastic Net}

  \begin{itemize}
    \item<1-> Proposed by \citet{zou2005}.
    \item<2-> Lasso (\(\ell_1\)) part:
          \begin{itemize}
            \item Enables sparsity (interpretability, parsimony, feature selection)
            \item Efficient when \(p \gg n\) (due to screening rules~\citep{elghaoui2010,tibshirani2012})
          \end{itemize}
    \item<3-> Ridge (\(\ell_2\)) part
          \begin{itemize}
            \item Mitigates lasso issue in correlated data
            \item Better predictive performance when true signal is non-sparse
          \end{itemize}
          % \item<4-> Very efficient solvers for the full path (coordinate descent)
  \end{itemize}
\end{frame}

\end{document}

